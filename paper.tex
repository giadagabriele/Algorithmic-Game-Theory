\documentclass{article}
\usepackage[utf8]{inputenc}

\title{Algorithmic Game Theory - Assignment}
\author{Giada Gabriele}
\date{25 February 2022}

\begin{document}
\maketitle
\section{Rules}
\begin{itemize}
\Large{
\item set of users \textit{U}
\item set of locations \textit{L}
\item set of utilities \textit{$u_i$} for each user \textit{i}
\item set of maximum payments \textit{$m_i$} for each user \textit{i}
\item set of proposes of users \textit{$p_u$(L, $U_i$, $m_i$)}
\item set of distance of two locations \textit{distance($L_1$,$L_2$,d)}
\item set of vehicles \textit{v(l, k, maximum\_length)} where \textit{k} is the number of users that the vehicle can accommodate and \textit{maximum\_length} is the bound in km of the tour with that vehicle   
}
\item set of costs proportional to the kilometers \textit{C\_km}
\item set of fixed costs \textit{C\_fix}
\end{itemize}
\section{Goal}
\Large{
We have to select the user that will perform the tour, define the locations and the payments that are charged to the users. The only constraint that we have is the maximum payment. Costs proportional to kilometers should be fairly distributed and the fixed costs should be based on declared utilities, so first we have to implement a mechanisms that lead to truthfully declare the utilities.
}
\section{The "right" mechanism}
\section{Best outcome}
\section{Conclusion}
\end{document}