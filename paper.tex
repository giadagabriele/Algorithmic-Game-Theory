\documentclass{article}
\usepackage[utf8]{inputenc}

\title{Algorithmic Game Theory - Assignment}
\author{Giada Gabriele}
\date{25 February 2022}
\usepackage{amsmath}
\usepackage{amssymb}
\begin{document}

\maketitle
\section{Goal}
\large{
We have to select the user that will perform the tour, define the locations and the payments that are charged to the users. The only constraint that we have is the maximum payment. Costs proportional to kilometers should be fairly distributed and the fixed costs should be based on declared utilities, so first we have to implement a mechanisms that lead to truthfully declare utilities.
}
\section{Mechanism choice}
\large{
We start from the definition of \textbf{what} is a Mechanism:
\begin{quote}
    A \textit{Mechanism} (for a Bayesian game setting ($N,O,\Theta,p,u)$) is a pair (A,M), where
    \begin{itemize}
        \item A = $A_1 \times \dots A_n$, where $A_i$ is the set of actions available to agent $i \in N$; and
        \item M : $A \mapsto \Pi(O)$ maps each action profile to a distribution outcomes.
    \end{itemize}
\end{quote}
Like we said before we want to pay attention on the truthfulness of the game and we know that a \textit{direct mechanism} is one in which the only action to each agent is to announce his private information. Since in a Bayesian game an agent's private information is his type, direct mechanism have $A_i = \Theta_i$, so a direct mechanism is said to be truthful (or incentive compatible) if, for any type vector $\theta$, in equilibrium of the game defined by the mechanism every agent \textit{i}'s strategy is to announce his true type, so that $\hat{\theta}_i = \theta_i$.\newpage Now we can exploit the definition of \textbf{Revelation principle}:
\begin{quote}
    If there exists any mechanism that implements a social choice function C in dominant strategies then there exists a direct mechanism that implements C in dominant strategy and is truthful.
\end{quote}
Given the revelation principle we can restrict our attention to truthful mechanism, the one that, for me, is more suitable is \textit{Quasilinear preferences}:
\begin{quote}
    Agents have quasilinear preferences in a \textit{n}-player Bayesian game when the set of outcomes is
    \begin{center}
        $O = X \times \mathbb{R}^n$
    \end{center}
    for a finite set X, and the utility of an agent \textit{i} given joint type $\theta$ is given by
    \begin{center}
        $u_i$(o,$\theta$) = $u_i$(x,$\theta$) - $f_i$($p_i$),
    \end{center}
    where $o = (x,p)$ is an element of $O, u_i : X \times \Theta \mapsto \mathbb{R}$ is an arbitrary function  and $f_i : \mathbb{R} \mapsto \mathbb{R}$ is a strictly monotonically increasing function.
\end{quote}
Then we split the mechanism into two pieces that are linearly related:
\begin{center}
    $u_i(o,\theta) = u_i(x,\theta) - f_i(p_i)$
    \begin{itemize}
        \item $x \in X$ is a discrete non-monetary outcome
        \item $p_i \in R$ is a monetary payment that agent \textit{i} must make to the mechanism
    \end{itemize}
\end{center}
If we assume that agents' preferences are quasilinear we are in a setting in which the agents' degree of preferences for the selection of any choice $x \in X$ is independent from his degree of preference for having to pay the mehcanism some amount $p_i \in \mathbb{R}$. In this way agents care only about the choice selected and about their own payments, not about the monetary payments made by other agents; we have defined fixes set of agents.\newpage At this point we can define a \textbf{Quasilinear mechanism}:
\begin{quote}
    A mechanism in the quasilinear setting (for a Bayesian game setting($N,O = X \times \mathbb{R}^n,\Theta,p,u$)) is a triple ($A,\chi,\wp$), where
    \begin{itemize}
        \item $A = A_1 \times \dots A_n$, where $A_i$ is the set of actions available to agent $i \in N$
        \item $\chi : A \mapsto \Psi(X)$ maps each action profile to a distribution over choices, and
        \item $\wp : A \mapsto \mathbb{R}^n$ maps each action profile to a payment for each agent.
    \end{itemize}
\end{quote}
We have split the function M in two functions, $\chi$ is a \textbf{choice rule} and $\wp$ is the \textbf{payment rule}. Now we exploit the \textbf{direct quasilinear mechanism} which is the only setting where each agent is asked to state his type.
\begin{quote}
    A direct quasilinear mechanism (for a Bayesian game setting($N,O = X \times \mathbb{R}^n,\Theta,p,u$)) is a pair ($\chi,\wp$); it defines a standard mechanism in the quasilinear setting, where each \textit{i}, $A_i = \Theta_i$.
\end{quote}
At this point we make again assumption that agents' utilities depend only on their own types exploiting the property of \textbf{conditional utility independence}:
\begin{quote}
    A Bayesian game exhibits conditional utility independence if for all agent $i \in N$, for all outcomes $o \in O$ and all pairs of joint types $\theta$ and $\theta' \in \Theta$ for which $\theta_i = \theta'_i$ it holds that $u_i(o,\theta) = u_i(o,\theta')$. 
\end{quote}
And again we go back at the definition of \textbf{Truthfulness}, in this case for a quasilinear mechanism:
\begin{quote}
    A quasilinear mechanism is truthful if it is direct and $\forall _i \forall v_i$, agent \textit{i}'s equilibrium strategy is to adopt the strategy $\hat{v}_i = v_i$.
\end{quote}
}
\end{document}