\documentclass{article}
\usepackage[utf8]{inputenc}

\title{Algorithmic Game Theory}
\author{Giada Gabriele}
\date{February 25, 2022}
\usepackage{amsmath}
\usepackage{amssymb}
\begin{document}

\maketitle
\section{Assignment goal}
\large{
The system has to select the user that will perform the tour; it has to define the locations that will be visited by the tour; and it has to define the payments that are charged to the users.
}
\section{Mechanism choice}
\large{
We start from the definition of \textbf{what} is a Mechanism:
\begin{quote}
    A \textit{Mechanism} (for a Bayesian game setting ($N,O,\Theta,p,u)$) is a pair (A,M), where
    \begin{itemize}
        \item A = $A_1 \times \dots A_n$, where $A_i$ is the set of actions available to agent $i \in N$; and
        \item M : $A \mapsto \Pi(O)$ maps each action profile to a distribution outcomes.
    \end{itemize}
\end{quote}
We want to pay attention on the truthfulness of the game and we know that a \textit{direct mechanism} is one in which the only action to each agent is to announce his private information. Since in a Bayesian game an agent's private information is his type, direct mechanism have $A_i = \Theta_i$, so a direct mechanism is said to be truthful (or incentive compatible) if, for any type vector $\theta$, in equilibrium of the game defined by the mechanism every agent \textit{i}'s strategy is to announce his true type, so that $\hat{\theta}_i = \theta_i$.\newpage Now we can exploit the definition of \textbf{Revelation principle}:
\begin{quote}
    If there exists any mechanism that implements a social choice function C in dominant strategies then there exists a direct mechanism that implements C in dominant strategy and is truthful.
\end{quote}
Given the revelation principle we can restrict our attention to truthful mechanism, the one that, for me, is more suitable is \textit{Quasilinear preferences}:
\begin{quote}
    Agents have quasilinear preferences in a \textit{n}-player Bayesian game when the set of outcomes is
    \begin{center}
        $O = X \times \mathbb{R}^n$
    \end{center}
    for a finite set X, and the utility of an agent \textit{i} given joint type $\theta$ is given by
    \begin{center}
        $u_i$(o,$\theta$) = $u_i$(x,$\theta$) - $f_i$($p_i$),
    \end{center}
    where $o = (x,p)$ is an element of $O, u_i : X \times \Theta \mapsto \mathbb{R}$ is an arbitrary function  and $f_i : \mathbb{R} \mapsto \mathbb{R}$ is a strictly monotonically increasing function.
\end{quote}
Then we split the mechanism into two pieces that are linearly related:
\begin{center}
    $u_i(o,\theta) = u_i(x,\theta) - f_i(p_i)$
    \begin{itemize}
        \item $x \in X$ is a discrete non-monetary outcome
        \item $p_i \in R$ is a monetary payment that agent \textit{i} must make to the mechanism
    \end{itemize}
\end{center}
In this way agents care only about the choice selected and about their own payments, not about the monetary payments made by other agents.\newpage At this point we can define a \textbf{Quasilinear mechanism}:
\begin{quote}
    A mechanism in the quasilinear setting (for a Bayesian game setting($N,O = X \times \mathbb{R}^n,\Theta,p,u$)) is a triple ($A,\chi,\wp$), where
    \begin{itemize}
        \item $A = A_1 \times \dots A_n$, where $A_i$ is the set of actions available to agent $i \in N$
        \item $\chi : A \mapsto \Pi(X)$ maps each action profile to a distribution over choices, and
        \item $\wp : A \mapsto \mathbb{R}^n$ maps each action profile to a payment for each agent.
    \end{itemize}
\end{quote}
We have split the function M in two functions, $\chi$ is a \textbf{choice rule} and $\wp$ is the \textbf{payment rule}. Now we exploit the \textbf{direct quasilinear mechanism} which is the only setting where each agent is asked to state his type.
\begin{quote}
    A direct quasilinear mechanism (for a Bayesian game\\ setting($N,O = X \times \mathbb{R}^n,\Theta,p,u$)) is a pair ($\chi,\wp$);\\ it defines a standard mechanism in the quasilinear setting, where each \textit{i}, $A_i = \Theta_i$.
\end{quote}
At this point we make again assumption that agents' utilities depend only on their own types exploiting the property of \textbf{conditional utility independence}:
\begin{quote}
    A Bayesian game exhibits conditional utility independence if for all agent $i \in N$, for all outcomes $o \in O$ and all pairs of joint types $\theta$ and $\theta' \in \Theta$ for which $\theta_i = \theta'_i$ it holds that $u_i(o,\theta) = u_i(o,\theta')$. 
\end{quote}
And again we go back at the definition of \textbf{truthfulness}, in this case for a quasilinear mechanism:
\begin{quote}
    A quasilinear mechanism is truthful if it is direct and $\forall _i \forall v_i$, agent \textit{i}'s equilibrium strategy is to adopt the strategy\\ $\hat{v}_i = v_i$.
\end{quote}
Also we would like to have a mechanism that is \textbf{efficient} \\(quasilinear mechanism with a selected choice x such that,\newpage $\forall v \forall x'$, $\displaystyle \sum_{i} v_i > \displaystyle \sum_{i} v_i(x')$) and \textbf{budget balanced} (quasilinear mechanism when $\forall v,\displaystyle\sum_{i} \wp_i (s(v))$), but we know, according to what game theory says, that a mechanism dominant-strategy incentive compatible, budget balance and efficient, doesn't exists. We have to relax on of them, it means that now we are talking about \textit{weak budget balance}:
\begin{center}
    $\forall v, \sum_{i} \wp_i(s(v)) \ge 0$
\end{center}
Now we can look at what is \textbf{cost sharing} and in particular we can look at a special case of the \textit{Shapley value} from a coalitional game theory that is the \textbf{the Shapley value mechanism}. This leads to an algorithm that is useful for modeling the problem, for computing the allocations and payments:
\begin{quote}
    $S \leftarrow N^*$\\
    \textbf{repeat}\\
    $\vert$ Find multicast routing tree T(S)\\
    $\vert$ Compute payments $p_i$ s.t. each agent \textit{i} pays an\\
    $\vert$ equal share of the cost for every link in T(\{i\})\\
    $\vert$ \textbf{foreach} $i \in S$ \textbf{do}\\
    $\vert$ \qquad \textbf{if} $\hat{v}_i < p_i$ \textbf{then} $S \leftarrow S$ \textbackslash \{i\}\\
    \textbf{until} no agents were dropped from S
\end{quote}

}
\end{document}